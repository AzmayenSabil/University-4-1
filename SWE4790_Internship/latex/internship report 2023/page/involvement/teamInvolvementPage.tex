\setcounter{section}{0} 

\begin{flushleft}
\textcolor[rgb]{0,0.690,0.941}{\textit{Role: Full-Stack Developer}}\\
\vspace{5pt}
In the internship my role was full-stack developer. So I worked with
the web development team. The web development team consisted of 10 engineers. 3 engineers from the frontend team, 3 engineers from the backend team, 2 engineers from the design team and 2 team
leader. I worked as a full-stack developer on the following project.\\
\vspace{5pt}
\begin{itemize}
    \item  \textbf{Ebidyloy project}\\
    \vspace{2pt}
    I spent most of my time working on this project. This is a public application and you can
access the website from this link: Ebdiyaloy. This is an EdTech-based application.
Ebidyloy project is available on android, ios, and web platforms. There are near about 10
engineers working on this project. I worked in the web team as a front-end developer.
    
\end{itemize}

The company usually asked the interns about tools and technologies they preferred to work
with. Because I already had experience working with Javascript, they assigned me to a team
who works with Javascript-based frameworks.

\end{flushleft}


\section{Overview of the team}
\begin{flushleft}
    The Ebidyloy team consists of near about 10 highly skilled and experienced engineers. As
Ebidyaloy is available on Android, iOS, and Web platforms, there is multiple sub-team in this
project to handle work on those platforms. I mainly worked on the web team as a front-end
developer. A breakdown of all team members is given below
\begin{itemize}
    \item Frontend Team (3 Members)
\item Backend Team (2 Members)
\item Dev-Ops Team (1 Member)
\item Design Team (1 Member)
\item QA Team (3 members)
\end{itemize}

There were also some support engineers in the team who are not permanent members but
sometimes work in different teams in case of any emergencies.
I worked as a support engineer in this project to help the team to migrate the website to a
different technology as well as fix some front-end issues and bugs.
\end{flushleft}

\section{My Influence as an Intern}
\begin{flushleft}
    Although I was an intern at DreamOnline Limited, the company always treat me like a full-time
employee. They always respected my opinions, wanted to hear what I want to say, and provide
me with constructive criticism about my work. As a member of the team, I always had to fulfill
several responsibilities -

\begin{itemize}
    \item Solve my assigned tasks.
 \item Give feedbacks to my team members' work
 \item Review UI and Ux
 \item Report any issue and bug that I can find in the project
 \item Join all team meetings and discussions on time
\end{itemize}

\end{flushleft}

\section{My contributions in Ebidyaloy Project}
\begin{flushleft}
\begin{itemize}
    \item \textbf{Implement Features and Functionalities}\\
    \vspace{6pt}
So basically Ebidyaloy project has three modules. One for students, one for teachers,
and one for management. During my stay in the company, I and my team fully developed
all the UIs and implemented all functionalities of the management module including
authentication and authorization workflow which has more than 100 pages.
\end{itemize}
\begin{itemize}
    \item \textbf{Example of My work}\\
    Some of the functionalities I developed include
    \begin{itemize}
        \item CRUD (Create/Read/Update/Delete) operations of the class Scheduling system.
Basically, users can do all the operations related to the schedule on the website
\item CRUD operations of the attendance system. Add the functionalities so that a
student can give his/her attendance, teachers can take attendance and
management can have all the records of attendance.
\item Routine uploading in CSV or Excel format as well as creating custom dynamic
routines within the website
\item Setup class and sections of a class
\item OTP registration and verification
\item Handling push notifications
\item CRUD operations of class-related notices
\item Handling class enrollment
\item Create a custom carousel. This carousel component is independent and reusable
    \end{itemize}



\item \textbf{Bug and Issue Fixing}\\
\vspace{6pt}
I solved a lot of bugs and issues that were encountered by my manager and QA team.
During this, I faced a lot of hurdles. But solving these bugs and issues makes me more
confident as a developer.
\item \textbf{Review UI/Ux}\\
\vspace{6pt}
I need to constantly review UIs and functionalities that my other team members
developed. If issues were found, I need to inform them about those issues as well as
give them feedback about their work.
\item \textbf{Support Team Members}\\
\vspace{6pt}
I also help my team to solve some of the issues and bugs they were facing. At the same
time provide feedback on their work.
\item \textbf{Research and Development}\\
\vspace{6pt}
For implementing some of the functionalities, I had to go through a research and
development process. For example, at a point, I need to implement a carousel in the
Ebidyaloy project. But because of the custom design, I can’t just directly use a 3rd party
package for implementing the carousel. So to implement a carousel in the project, I had
to first learn how a carousel works under the hood, and then I incorporated those
techniques in the project to implement the carousel. At the same time, I need to make
this carousel component reusable and customizable so that I and my team members can
use it when the need comes.\\
\end{itemize}
\vspace{6pt}
The majority of my time has been spent on the Ebidyloy project. I learned a lot of things working
on a production-level application. Solving a lot of real-life business problems helps me to
expand my knowledge and strengthen my front-end skills.\\

\subsection{Used Technologies}
\begin{flushleft}
    For the frontend development I have used React.js and
Tailwind CSS. For API documentation and API testing before fetching it on the front end I
used Swagger. And for the Version control system and project management, I used
Backlog. 
\end{flushleft}



\vspace{15pt}

% \textbf{Project Link :} \href{https://portal.dreamonlinelimited.xyz/login}{\textcolor[rgb]{0,0.690,0.941}{\textit{Portal}}}

\textbf{Project Link:} \href{https://portal.dreamonlinelimited.xyz/login}{\textit{Portal}}



\end{flushleft}


\section{Achievements}
\begin{flushleft}
    During my stay in the company, I worked on two projects. I gave my best to complete all the
tasks that were assigned to me. And because of my hard work, dedication, and skill, the
company offered me to join the company as a Junior Software Developer. Unfortunately, I had to
decline the offer as I still have to complete my academics. But the company kept their door open
for me so that I can join the company after my graduation of I want.

\end{flushleft}



