\setcounter{section}{0} 
\section{Technical Learning}
\begin{flushleft}
As I mainly worked as a frontend engineer, the tools and languages I learned and used are
primarily frontend focused.
\begin{itemize}
    \item \textbf{JavaScript:}\\
    \vspace{6pt}
JavaScripy is a lightweight, interpreted, and promise-based language. It is also called
the language of the web. Both projects that I have worked on use JavaScript.
\item \textbf{React.js:}\\
\vspace{6pt}
React is a JavaScript library that is used to build user interfaces. It is kept up-to-date by
Meta (previously Facebook) and a group of independent programmers and businesses.
\item \textbf{Next.js:}\\
\vspace{6pt}
This is a react-based framework for production applications. Next.js is used to build the
client side or frontend part of the Ebidyloy project.
\item \textbf{Tailwind CSS:}\\
\vspace{6pt}
This is a utility-first CSS framework for styling User Interfaces. I heavily used Tailwind
CSS while working on the Ebidyaloy project.
\item \textbf{Swagger:}\\
\vspace{6pt}
Swagger is used to test the APIs that were provided by the backend team. Before using
the APIs on the client side, I had to test the APIs and if something goes wrong or if I find
any issues, I need to report those to the backend team.
\item \textbf{Backlog:}\\
\vspace{6pt}
The company uses backlog for project management, bug, and issue tracking as well as
for version control management. During my stay at the company, I thoroughly used this
tool in my day-to-day work.
\end{itemize}



\end{flushleft}

\section{Soft skills learning}
\begin{flushleft}
Working in the company not only made me a better developer but also helps me to grow as a
better person.
\begin{itemize}
    \item \textbf{Collaborative/ Teamwork:}\\
    \vspace{6pt}
The most important thing I learned from this internship would be working in a team.
Although in university I also worked in a team to develop many projects but working in
the industry as a team is completely different from working on a university project. Like I
worked as a frontend developer in the team. But at the same time, I had to constantly
communicate with the backend team for any API-related issues and with the design
team for any changes coming from the client or any UI elements that need modification,
and also with the Quality Assurance (QA) Team regarding any issues and bugs. Apart
from that, I also need to communicate with my team members while solving critical
problems or implementing any new functionalities. All these things make me a better
team player as a whole.
\item \textbf{Time Management:}\\
\vspace{6pt}
The company is pretty strict about time management. The management encourages the
employees to maintain time carefully. Like during our morning scrum meeting, we need
to join the meeting at a specific time otherwise we are marked as absent. During the
product release period, we sometimes had to work overtime and even had to work on
weekends to meet the deadline.
\item \textbf{Critical Thinking:}\\
\vspace{6pt}
I worked on a production-level application during my internship, so when I try to solve a
particular problem in the project I need to solve it in the most efficient and optimized way
possible so that when a huge number of people come to the website, the website
shouldn’t face any kind of latency or performance issues. At the same time, my
implementation should be reusable and easily understandable so that if I or my team
members need to use the same implementation or functionality in any other place, they
can easily use it without any issues. I believe solving a lot of real-life business problems
makes me a better developer than I was before.
\item \textbf{Problem Solving:}\\
\vspace{6pt}
As a result of my internship experience working on a production-level program, I
need to handle each problem I have in the project in the most effective and
efficient way I can to improve the user experience. Likewise, my implementation
should be reusable and simple to comprehend so that if I or my team members
need to use the same implementation or feature somewhere else, they can do so
without any problems. I consider myself a better developer now than I ever was
before because I've dealt with many actual business logics
\item \textbf{ Accountability:}\\
\vspace{6pt}
If I implement a functionality or develop a UI and if it breaks or something goes wrong in
production, then it’s my responsibility to come forward, accept my mistake, and solve
that mistake. I just can’t push the mistake to my superiors or other team members to
solve this for me. I think it makes me more accountable for my work at the same time
makes me diligent in my work.
\item \textbf{ Patience:}\\
\vspace{6pt}
Throughout my internship journey, I have had to spend a huge amount of time for solving
business problems in the most optimal way possible as well as solving lots of bugs and
issues. And of course, the process wasn’t easy. In order to find a solution, I had to cope
with a lot of challenges and impediments. This process taught me that the most
important person you’ll ever have to be patient with is you.  
\end{itemize}
  
\end{flushleft}

\section{Additional Learning}
\begin{flushleft}
    The technologies that have been used in the projects I already knew all these. So my mentor
asked me to learn some additional topics which were not directly related to my work, but some
of the things that I should know as a software developer.
\begin{itemize}
    \item \textbf{Database Design:}\\
    \vspace{6pt}
During my last month of internship, my mentor showed me the database design of the
Ebidaloy project. Up until then, I only read about database design. But seeing how
developers design a production-level application helped me to understand database
design more precisely.
\item \textbf{Nginx}\\
\vspace{6pt}
Nginx is an open-source web server software that serves as a reverse proxy, HTTP load
balancer, and email proxy for IMAP, POP3, and SMTP. The company used Nginx in the
project as a web server. Similar to docker, I only read about how Nginx works but did not
use it practically.
\item \textbf{ Proxy \& Reverse Proxy}\\
\vspace{6pt}
A reverse proxy proxies on behalf of servers, whereas forward proxy or just proxy
proxies on behalf of clients (or requesting hosts). Basically, a reverse proxy receives
requests from external clients on behalf of the servers located behind it.
\end{itemize}



\end{flushleft}




