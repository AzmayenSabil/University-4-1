\setcounter{section}{0} 
\section{Introduction}
\begin{flushleft}
This document provides a detailed report of my internship experience at DreamOnline Limited.
From the 9th of May to the 9th of September, I worked as a full-stack development intern at
DreamOnline Limited. Because of this internship, I get to know how a production-level
application with hundreds of users is built. Throughout our academic life, we only get theoretical
knowledge. But to sustain in a competitive world and become a good engineer, having practical
knowledge is a must. This internship expose me to various tools and technologies of software
development and makes me a better software engineer than I was before.
During my stay in the company, I worked on two production-level applications. One is the
Ebidyaloy project which is an EdTech-related product and one is the WebGym project. My main
goal was to develop software applications following all the stages of the software development
life cycle through agile scrum methodologies. In those projects, I implemented a good number of
features and functionalities and solved a lot of bugs. I also regularly attended all my scrum
meetings as well as other official meetings. Throughout my internship, I had to deal with many
hurdles, but I was able to successfully solve all of those. In the company, I worked with many
trending tools and technologies. I learn the best practices and many clean code methodologies
from working on real-life projects. Working on these projects not only helped me to increase my
skills but also makes me more confident as a software engineer.
This internship has taught me the various aspects and practices of software engineering in the
real world, professional development in scrum methodology and helped me enhanced my
technical abilities. I had a chance to put all concepts and knowledge acquired at the university
into good professional practice. The experience of working in a team makes me more
considerate of others’ opinions and helps me to become a good team player. I also get a
perspective of working in the business world. This gives me the chance to broaden my
knowledge and identify my strengths and weaknesses, both of which will be useful for my future
profession

\vspace{10pt}

\end{flushleft}

\section{Background}
\begin{flushleft}
    I started my internship on the 9th of May, 2022. Although my internship was for a full-stack
development intern, I mainly worked as a front-end developer in the company. On the first day of
the internship, the company asked for my choice of technology and assigned me a team. During
my first 1 week of internship, the team lead asked me to go through the frontend and backend
code of the project and understand the workflow of the project as I already knew the
technologies that were used in the project. After 1 week, I was assigned to the front-end
sub-team, and tasks were assigned to me by the front-end team lead.
I was also assigned to work on another project as a front-end developer as I already performed
really well on my first project. I worked 2 to 3 weeks on that project and help the team to solve
some issues and bugs they were facing on that project as the team had no dedicated front-end
developer when I joined.
In my internship, I got the opportunity to work with the technologies I loved the most. That is
Next.js. I also used tailwind CSS and swagger in my day-to-day task
\end{flushleft}


\section{Motivation}
\begin{flushleft}
    The internship program gives me the chance to broaden my skill set, gain experience working in
the industrial sector, and achieve my aim of becoming accustomed to the corporate
environment. This internship gives me an opportunity to collaborate with a company and helps
me establish a connection with them. I can hone my skills and broaden my experience through
this involvement. I have the opportunity to learn a variety of new and trending tools and
technologies. I've always wanted to contribute significantly to a production-level application and
solve real-life business problems. And I'm committed to doing my absolute best to accomplish
these. I was able to gain the necessary information about the software industry through this
internship program
\end{flushleft}


\section{Objective}
\begin{flushleft}
    The goal of this report is to illustrate my internship-related experiences. I have used my
academic knowledge for solving real-world problems, learnt about the corporate work culture
and other best practices to cope with the corporate world. The key objectives that were focused
on during my internship are
1. Getting used to the office environment.
2. The concepts, tools, and technologies I have learned will be helpful to me in the future.
3. In the future, I will benefit from knowing the working environment and culture.
4. Key characteristics that I need to improve in order to adapt to the job market.
5. Take into account the needs of the current and upcoming industries.
The main objective of the internship is to solve business problems, to know how a production
level is built from scratch, and to gather information on the industry environment. As an intern,my superior assigned me tasks. The tasks were about feature implementation, bugs, and issue
fixing as well as some research and development work. I frequently communicate with my
manager and team lead to know about my progress and asked them to provide feedback on my
work. I regularly attend all the meetings and followed the company’s policy strictly.

\end{flushleft}

\section{Scope}
\begin{flushleft}
    The primary goal of the internship was to become a good engineer. Throughout my internship, I
learned a lot of things. I was able to connect the dots between my academic learning and
industrial learning. If I want to sum up my learning in the internships in some points:
● Communication is the key when it comes to working in a team
● Coding this not everything when developing a product. There are a lot of things we need
to consider when developing a product
● Have to follow the rules and policies when working in a reputed company
● When solving a particular problem in a production-level application we have to do it in
the most optimized and efficient way possible
● Never stop learning. We need to continuously learn to grow as a developer.
\end{flushleft}
